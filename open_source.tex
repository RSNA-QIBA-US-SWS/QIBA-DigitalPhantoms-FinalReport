\section{Open-Source Numerical Shear Wave Solvers}\label{sect:open_source}

\subsection{FEBio}
Building on Round III effort, Dr. Jiang has extended the FEBio (INSERT FOOTNOTE
URL) model of elastic shear wave propagation to include viscoelastic media.  An
example FEBio simulation template has been provided in the GitHub repository:

\url{https://github.com/RSNA-QIBA-US-SWS/QIBA-DigitalPhantoms/tree/master/febio}

Below is a comparison of the reconstructed viscoelastic material metrics
reconstructed for the 3 configurations used to generate the viscoelastic
digital phantoms:

INSERT TABLE HERE

It was found during Round III that the lack of a perfect matching layer
boundary condition or any other damping boundary that can simulate a
semi-infinite medium introduced appreciable artifact in the FEBio models.
These boundary condition limitations are still a first-order limitation of
using FEBio as a replacement for the commercial FEA solvers.

\subsection{C-Scan Finite Difference Code}
Dr. McAleavey has developed finite difference code solving shear wave
propagation in C-scan planes in viscoelastic media.  This code has been
optimized to run in a GPGPU environment on a desktop workstation and requires
considerably lower computation overhead than the commercial FEA software
packages used to generate the digital phantom data uploaded to QIDW.

Source code for this finite difference algorithm has been shared in the GitHub
repository:

INSERT HYPERLINK HERE

Below is a comparison of the reconstructed viscoelastic material metrics
reconstructed for the 3 configurations used to generate the viscoelastic
digital phantoms:

INSERT TABLE HERE

This code provides accurate solutions in the C-scan plane, but full volumetric
simulated displacement / velocity data are not yet available using this
approach.  This code will continue to evolve through external funding beyond
the scope of QIBA support.
